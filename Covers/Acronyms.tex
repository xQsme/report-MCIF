%----------------------------------------------------------------------------------------
%   Acronyms
%----------------------------------------------------------------------------------------

\newacronym{dns}{DNS}{Domain Name System}
\newacronym{adsl}{ADSL}{Assimetric Digital Subscriber Line}
\newacronym{ascii}{ASCII}{American Standard Code for Information Interchange}
\newacronym{bios}{BIOS}{Basic Input/Output System}
\newacronym{bit}{bit}{Digito binário}
\newacronym{byte}{Byte}{Unidade de informação digital composta por oito bits}
\newacronym{cpu}{CPU}{Central Processing Unit}
\newacronym{codec}{CODEC}{COmpression/DECompression}
\newacronym{dll}{DLL}{Dynamic Link Library}
\newacronym{dhcp}{DHCP}{Dynamic Host Configuration Protocol}
\newacronym{ftp}{FTP}{File Transfer Protocol}
\newacronym{ip}{IP}{Internet Protocol}
\newacronym{isp}{ISP}{Internet Service Provider}
\newacronym{so}{SO}{Sistema Operativo}
\newacronym{tcp}{TCP}{Transmission Control Protocol}
\newacronym{abc}{ABC}{A lista de acrónimos deve ficar ordenada alfabeticamente}


%----------------------------------------------------------------------------------------
%   Formatação da página dos acrónimos
%----------------------------------------------------------------------------------------
\renewcommand{\acronymname}{List of Abbreviations}
% \markboth{\spacedlowsmallcaps{\acronymname}}{\spacedlowsmallcaps{\acronymname}}
\refstepcounter{dummy}
\phantomsection 
\addcontentsline{toc}{chapter}{\acronymname}
% \addtocontents{toc}{\protect\vspace{\beforebibskip}} % to have the bib a bit from the rest in the toc

%  serve para eliminar quebras de página a mais
\begingroup
\let\clearpage\relax
\let\cleardoublepage\relax
\let\cleardoublepage\relax


\glsaddall

\printglossary[type=\acronymtype,style=super,title=\acronymname]

% as definições de abreviaturas estão no ficheiro
% 	Covers/Acronyms-list.tex


\endgroup	

\cleardoublepage
