%----------------------------------------------------------------------------------------
%   Acronyms
%----------------------------------------------------------------------------------------

\newacronym{tsk}{TSK}{The Sleuth Kit}
\newacronym{ftk}{FTK}{Forensics Toolkit}
\newacronym{os}{OS}{Operating System}
\newacronym{jwt}{JWT}{JSON Web Token}
\newacronym{uuid}{UUID}{Universally Unique Identifier}
\newacronym{rsa}{RSA}{Rivest-Shamir-Adleman}
\newacronym{otp}{OTP}{One Time Password}
\newacronym{u2f}{U2F}{Universal Second Factor}
\newacronym{nist}{NIST}{National Institute of Standards and Technology}
\newacronym{css}{CSS}{Cascading Style Sheets}
\newacronym{dos}{DOS}{Disk Operating System}
\newacronym{bsd}{BSD}{Berkeley Software Distribution}
\newacronym{gpt}{GPT}{GUID Partition Table}
\newacronym{ascii}{ASCII}{American Standard Code for Information Interchange}
\newacronym{gpt}{GPT}{GUID Partition Table}
\newacronym{ntfs}{NFTS}{New Technology File System}
\newacronym{stix}{STIX}{Structured Threat Information eXpression}
\newacronym{ftp}{FTP}{File Transfer Protocol}


%----------------------------------------------------------------------------------------
%   Formatação da página dos acrónimos
%----------------------------------------------------------------------------------------
\renewcommand{\acronymname}{Lista de Abreviaturas}
% \markboth{\spacedlowsmallcaps{\acronymname}}{\spacedlowsmallcaps{\acronymname}}
\refstepcounter{dummy}
\phantomsection 
\addcontentsline{toc}{chapter}{\acronymname}
% \addtocontents{toc}{\protect\vspace{\beforebibskip}} % to have the bib a bit from the rest in the toc

%  serve para eliminar quebras de página a mais
\begingroup
\let\clearpage\relax
\let\cleardoublepage\relax
\let\cleardoublepage\relax


\glsaddall

\printglossary[type=\acronymtype,style=super,title=\acronymname]

% as definições de abreviaturas estão no ficheiro
% 	Covers/Acronyms-list.tex


\endgroup	

\cleardoublepage
