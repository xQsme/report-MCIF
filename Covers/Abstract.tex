%*******************************************************
% Abstract
%*******************************************************


\renewcommand{\abstractname}{Resumo}
\markboth{\spacedlowsmallcaps{\abstractname}}{\spacedlowsmallcaps{\abstractname}}
\refstepcounter{dummy}
\addcontentsline{toc}{chapter}{\abstractname}


\begingroup
\let\clearpage\relax
\let\cleardoublepage\relax
\let\cleardoublepage\relax

\chapter*{Resumo}

No contexto do mestrado em Cibersegurança e Informática Forense, uma das escolhas para o último ano curricular é a realização de um estágio,
sendo este documento o relatório resultante da realização desse estágio.

A empresa \company \space em concordância com o estudante definiram um plano de trabalho, a realizar durante o decorrer do estágio curricular de 9 meses, 
com o objetivo de desenvolver uma plataforma de análise forense digital baseada na plataforma Autopsy.

Existem bastantes plataformas de análise forense digital, mas o Autopsy é a opção grátis e de código aberto com mais reconhecimento no mercado.

A plataforma desenvolvida, e sobre a qual incide este trabalho e por conseguinte este relatório, tem como objetivo complementar a plataforma Autopsy com uma das funcionalidades mais importantes das plataformas de análise forense digital,
a colaboração, adaptando a arquitetura da plataforma para um modelo cliente-servidor.

O desenvolvimento da plataforma decorreu com base nas práticas habituais da empresa, utilizando uma \emph{framework} ágil e trabalhando com diferentes entidades como \emph{designer}, \emph{tester} e \emph{product owner}.

\bigskip


\endgroup			

\cleardoublepage

\renewcommand{\abstractname}{Abstract}
\markboth{\spacedlowsmallcaps{\abstractname}}{\spacedlowsmallcaps{\abstractname}}
\refstepcounter{dummy}
\addcontentsline{toc}{chapter}{\abstractname}


\begingroup
\let\clearpage\relax
\let\cleardoublepage\relax
\let\cleardoublepage\relax

\chapter*{Abstract}

In the scope of the master's degree in Cybersecurity and Digital Forensics, one of the choices for the final curricular year was preforming an internship. 
This document is the report produced from the realization of said internship.

\company \space and the student reached an agreement for a 9 month long curricular internship, where the scope is building a digital forensics platform based on the existing Autopsy platform.

Plenty of options of digital forensics platforms are available, but Autopsy is the free and open source option with most recognition in the market.

The developed platform, which is documented by this report, aims to complement the existing Autopsy platform with one of the most important features digital forensics platforms have, 
which is collaboration, by adapting the platform's architecture into a client-server model.

The development of the platform was done with the usual practices of the company, by having an agile framework and working with different parties such as a designer, tester and product owner.

\bigskip


\endgroup           
