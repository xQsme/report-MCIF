\addtocontents{toc}{\protect\vspace{\beforebibskip}} % Place slightly below the rest of the document content in the table

%Background / Related Work

%Digital Forensics
%TSK
%Autopsy
%Alternative Software
%Proposed Solution

%************************************************
\chapter{Background}
\label{ch:background}
%************************************************

The scope of this internship concerns digital forensics, as it focuses on adapting an existing forensics platform into a collaborative client-server model.

In this chapter, a contextualization of digital forensics is given, an analysis of both \citetitle{sleuthkit} \cite{sleuthkit} and \citetitle{autopsy} \cite{autopsy} is made, it's given a brief description of
the existing forensic platform alternatives and is described the proposed solution for the scope of the internship.

\section{Digital Forensics}

Forensic science is the use of scientific methods or expertise to investigate crimes
or examine evidence that might be presented in a court of law.

The definition of digital forensics is directly related to the definition of
computer forensics which is the collection, preservation, analysis,
and presentation (\citetitle{daniels} \cite{daniels}) of evidence stemming from digital sources for use in a legal matter
using investigative processes, tools, and practices.

Digital forensics is the application of computer technology to criminal cases where evidence
includes items that are created by digital systems.

Digital forensics, according to \citeauthor{nist} \cite{nist}, is the field of forensic science that is concerned with retrieving,
storing and analyzing electronic data that can be useful in criminal investigations.
This includes information from computers, hard drives, mobile phones and other data
storage devices.

Digital forensic investigators face challenges such as extracting data from damaged or destroyed
devices, locating individual items of evidence among vast quantities of data,
and ensuring that their methods capture data reliably without altering it in any way.

Personal data should ultimately be attributable to an individual; however, making
that attribution can be difficult due to the presence or absence of individualized
user accounts, security to protect those user accounts, and the actual placement of a
person at the same location and time when the data is created.

\subsection{Digital Evidence}

Digital evidence is any type of digital data with incriminating characteristics,
which can result from any type of action preformed by a user, like transactions,
recordings, or virtually any action preformed on a device.

Nowadays it's virtually impossible not to leave a digital track behind, since most
of us carry and use devices capable of connecting to the internet.

The explosion of social media sites has created a whole new area of electronic
evidence. Most people today are willing to share all kinds of information through
social media platforms.

In order for electronic data to become digital evidence, it must be stored and be recoverable
by a forensic examiner. One of the great challenges is not whether digital
evidence may exist, but where the evidence is stored, getting access to that storage,
and finally, recovering and processing that digital evidence for relevance within
a civil or criminal action.

The potential storage options for electronic evidence has shifted from being only contained locally
to being either located locally or remotely in what is called ``The Cloud`` \cite{cloud}. 

More and more everyday computing processes are moving to the Internet
where companies offer software as a service. Software as a service means that
the customer no longer has to install software on their computer, allowing access to the software remotely,
and not storing any data locally.

%%Comment about difficulty between retrieving local evidence and cloud stored evidence

\subsection{Processes and Procedures}

Digital forensics is the application of forensic science to electronic evidence in a
legal matter.

While there are many different subdisciplines and many types of devices, communication,
and storage methods available, the basic principles of digital forensics
apply to all of them.

These principles encompass four areas:

\begin{enumerate}
\item Acquisition
\item Preservation
\item Analysis
\item Presentation
\end{enumerate}

Each of these areas includes specific forensic processes and procedures.

\subsubsection*{Acquisition}

Acquisition is the process of collecting electronic data. Seizing a computer at a crime scene or
taking custody of a smartphone in a civil suit are examples of device acquisition, but the data 
must be extracted from these devices using specific procedures that equate to making a copy of
the storage devices, while following strict rules to ensure the integrity of all the extracted data.

Since acquisition is the first interaction between the investigators and the evidence,
it is the step where it's most likely to occur modifications of the contents of the seized devices,
because turning on the device or extracting the data without following the right procedures can alter
it's contents irreversibly.

\subsubsection*{Preservation}

For evidence to be defendable in court it must be preserved properly.
Preservation in the forensics context is the process of creating a chain of custody
that begins before collecting the evidence and ends when the evidence is released.
Any interference in the chain of custody can lead to issues regarding the validity of the evidence.
Additionally, preservation includes maintaining the evidence in a safe environment, preventing
intentional destruction with malicious purpose or accidental modification by unqualified people.

A chain of custody log allows proving that the integrity of the evidence has been maintained from seizure
through presentation in court. It should contain entries for every time that a piece of evidence
has been touched, including collection, storage transport, and any time the evidence is checked out for
handling by any personnel.

\subsubsection*{Analysis}

Analysis is the process of locating and categorizing items from evidence
that has been collected in a case. Each case is unique as the
circumstances surrounding each case can vary immensely, not only in the evidence being analyzed, but also in the
approach used to perform the analysis. The analysis is the area where
the individual skills, tools used, and the training of the forensic examiner have
the greatest impact on the outcome of the examination. Considering that electronic evidence
appears in so many forms and comes from diverse locations and devices,
the training and experience of the examiner has a much greater impact
on the results of the examination.

Analysis of digital evidence is more than just determining whether a file exists on a hard drive, it involves finding out how
that file got on the hard drive, and if possible, who put the file on the hard drive.

\subsubsection*{Presentation}

Presentation of the examiner's findings is the last step in the process of forensic
analysis of electronic evidence. This includes not only the written findings or forensic
report, but also the creation of sworn statements, depositions of experts, and court testimony.
There are no exact rules or standards for reporting the results of an
examination. Each entity may have its own particular guidelines
for reporting. However, forensic examination reports should be written clearly, concisely,
and accurately, explaining what was examined, the tools used for the examination,
the procedures used by the examiner, and the results of the examination.
The report should also include the collection methods used, including specific steps
taken to protect and preserve the original evidence and how the verification of the
evidence was performed.

\section{The Sleuth Kit}

The Sleuth Kit (TSK) is a library and collection of command line tools that allow the
investigation of disk images. The core functionality of TSK allows volume and file system data analysis.
The plug-in framework allows incorporation of additional modules to analyze file contents
and build automated systems. The library can be incorporated into larger digital forensics tools and
the command line tools can be directly used to find evidence.

\subsection{Overview}

The original part of Sleuth Kit is a C library and collection of command line
file and volume system forensic analysis tools. The file system tools allows examining file systems
of a computer in a non-intrusive fashion. Because the tools do not rely on the operating system to process
the file systems, deleted and hidden content can be shown.

The volume system tools allow examination of the layout of disks
and other media. TSK supports DOS partitions, BSD partitions, Mac partitions, Sun slices, and 
GPT disks. With these tools, partition locations can be identified 
and extracted so that they can be analyzed with file system analysis tools.

When performing a complete analysis of a system, command line 
tools can become tedious. Autopsy is a graphical interface to the tools in TSK, 
which allows easier conduction of an investigation. Autopsy 
provides case management, image integrity, keyword searching, and other automated
operations.

A complete analysis also requires more than just file and volume system analysis.
However, a single tool can't provide support for all file types and analysis 
techniques. The TSK Framework allows tools to easily incorporate file analysis
modules that were written by other developers.

TSK Analyzes raw, Expert Witness and AFF file system and disk images.
It supports the NTFS, FAT, ExFAT, UFS 1, UFS 2, EXT2FS, EXT3FS, Ext4, HFS, ISO 9660,
and YAFFS2 file systems.

\subsection{Search Techniques}

TSK allows listing allocated and deleted ASCII and Unicode file names, can display the
details and contents of all NTFS attributes, can display file system and meta-data structure details,
can create time lines of file activity, which can be imported into a spread sheet to create graphs and reports.
TSK allows the lookup of file hashes in hash databases, it organizes files based on their type, and pages of
thumbnails can be made from graphic images to facilitate quick analysis.

\section{Autopsy}

Autopsy is a digital forensics platform and a graphical interface to The Sleuth Kit
along with other digital forensics tools. It is used by law enforcement, military, 
and corporate examiners to investigate what happened on a computer. It can even 
be used by anyone to recover photos from a camera's memory card.

Autopsy was designed to be intuitive out of the box. Installation is easy and
wizards guide the user through every step. All results are shown in a single tree.

Autopsy was designed to be an end-to-end platform with modules that come with
it out of the box and others that are available from third-parties. Some of the
modules provide:

\begin{table}[h]
  \begin{tabularx}{\textwidth}{@{}|c| *1{>{\centering\arraybackslash}X}@{}|}
    \hline
    \textbf{Name} & \textbf{Description} \\
    \hline\hline
    Timeline Analysis & Advanced graphical event viewing interface \\
    \hline
    Hash Filtering & Flag known bad files and ignore known good \\
    \hline
    Keyword Search & Indexed keyword search to find files that mention relevant terms \\
    \hline
    Web Artifacts & Extract history, bookmarks, and cookies from Firefox, Chrome, Internet Explorer and Microsoft Edge \\
    \hline
    Data Carving & Recover deleted files from unallocated space \\
    \hline
    Multimedia & Extract EXIF metadata from pictures and videos and display these files \\
    \hline
    Indicators of Compromise & Scan a computer using STIX \\
    \hline
  \end{tabularx}
\end{table}

Autopsy runs background tasks in parallel using multiple cores and provides results as soon as they are found.
It may take hours to fully search a drive, but the user will know in minutes if certain keywords were found in a specific folder.

Autopsy is free. As budgets are decreasing, cost effective digital forensics solutions are essential. Autopsy offers
the same core features as other digital forensics tools and offers other essential features, such as web artifact analysis
and registry analysis, that other commercial tools do not provide.

\section{Related Software}

\subsection{Nuix}

\citetitle{nuix} \cite{nuix} allows investigators to work efficiently on gigabyte to terabyte-sized investigations and beyond.
It’s ideal for local or small regional forensic labs struggling with the expanding volume, variety, and 
complexity of digital evidence and looking to build or upgrade a dedicated digital forensics facility.

If the resources are spread geographically or companies are looking to facilitate greater collaboration across departments,
Nuix Lab breaks down evidence silos and makes better use of existing team members and intelligence. Nuix software puts 
evidence into the hands of less technical reviewers or case officers sooner in the investigation.

The core technologies of the Nuix Lab, Nuix Workstation and Nuix Investigate, give digital forensic technicians and case investigators 
different lenses into the same case data. Investigators benefit from an easy-to-use browser experience where they can collaborate on 
the same data at the same time, creating efficiency and helping them share insights.

Implementing \citetitle{elasticsearch} \cite{elasticsearch} as a data store for the Nuix Lab boosts evidence processing, investigation, and intelligence capabilities. 
It’s appropriate for investigations that contain massive volumes of digital evidence and numerous digital exhibits; are conducted across multiple
regions or jurisdictions, or need to cross-reference and correlate intelligence across multiple current and historical cases. 

In addition, it contains powerful artificial intelligence, machine learning, and analytics to supercharge the investigations. 

\subsection{EnCase Forensic}

\citetitle{encase} \cite{encase} enables quickly searching, identifying, and prioritizing potential evidence, in computers and mobile devices, to determine whether 
further investigation is warranted. This will result in a decreased backlog so that investigators can focus on getting to case closed. 

EnCase Forensic helps acquire more evidence than any product on the market. Can collect from a wide variety of operating and file systems, 
including over 25 types of mobile devices with EnCase Forensic. Parses the most popular mobile apps across iOS, Android, and Blackberry devices so that 
no evidence is hidden. This is the flexibility needed to ensure teams can complete cases no matter where the potential evidence resides.

EnCase Forensic is unmatched in its decryption capabilities, offering the broadest support of any forensic solution. Encryption support includes products 
such as Dell Data Protection, Symantec, McAfee, and many more. Decryption power can be further expanded with Tableau Password 
Recovery — a purpose-built, cost-effective hardware solution to identify and unlock password-protected files. 

The EnCase Forensic evidence processor provides industry-leading processing capabilities that can automate the preparation of evidence, making 
it easier to complete an investigation. Powered by an indexing engine built for scale and performance, it can automate complex queries across 
varied evidence sources in one step saving time and increasing efficiency. 

The most important part of any investigation is your ability to analyze evidence. EnCase Forensic is built with the investigator in mind, 
providing a wide range of capabilities that enables performing deep forensic analysis as well as fast triage analysis from the same solution. 

EnCase Forensic provides a flexible reporting framework that empowers tailoring case reports to meet specific needs. With comprehensive 
and triage reporting options built in, can create reports for a wide range of audiences and easily share them across an organization. 

\subsection{Forensics Toolkit}

\citetitle{ftk} \cite{ftk}  is an award-winning, court-cited digital investigations solution built for speed, stability and ease of use. It quickly 
locates evidence and forensically collects and analyzes any digital device or system producing, transmitting or 
storing data by using a single application from multiple devices. Known for its intuitive interface, email analysis, 
customizable data views, processing speeds and stability.

All digital evidence is stored in one case database, giving teams access to the most current case evidence. It 
reduces the time, cost and complexity of creating multiple datasets. And best of all, there is continuous data transfer between AccessData's forensic 
and e-discovery solutions, allowing for true collaboration between all parties working on the case. 

With customizable processing, teams have the ability to establish enterprise-wide processing standards, creating 
consistency for the investigations and reducing the possibility of missed data. Since evidence is processed up 
front, investigators don't have to wait for searches to execute during the analysis phase. FTK is designed to provide the fastest, 
most accurate and consistent processing with distributed processing and true multi-threaded/multi-core support.

Indexing is done up front, so filtering and searching are faster than with any other solution. FTK offers 
the flexibility to perform multipass data review and change indexing options without reprocessing data. 
Whether teams are in the investigating phase or performing document review they have a shared index 
file, eliminating the need to recreate or duplicate the file. Most importantly, they receive consistent search 
results regardless of whether they are searching in FTK or Summation. Social Analyzer allows viewing email 
communications at the domain level and drill down to the custodian level to see communications among 
specific individuals. 

FTK allows users to create images, process a wide range of data types from forensic images to email archives and 
mobile devices, analyze the registry, crack passwords, and build reports—all within a single solution. 

With the single-node enterprise, users can preview, acquire and analyze evidence remotely from computers 
on the same network. 

Automatically constructs timelines and graphically illustrates relationships among parties of interest in a 
case. With Email, Social and File Visualization users can view data in multiple display formats, including timelines, 
cluster graphs, pie charts, geolocations and more, to help determine relationships and find key pieces 
of information. Then generate reports that are easily consumed by attorneys, CIOs or other investigators. 

Almost every investigation involves the analysis of Internet artifacts. Web browsing caches store records of sites a 
suspect has visited, web-based emails may help to prove intent or correlate other events and instant message 
conversations or social media sites can contain evidence. 
When evidence is processed, artifact files are categorized and organized so they can easily be seen. 

Available as an option to FTK, \citetitle{cerberus} \cite{cerberus} is an automated malware triage platform solution designed to integrate 
with FTK. It's a first layer of defense against the risk of imaging unknown devices and allows identifying 
risky files after processing data in FTK. Then users can see which files are potentially infected and can avoid 
exporting them. Cerberus is one tool in the malware arsenal and helps identify potentially malicious files.

Teams can:
\begin{itemize}
\item Determine both the behavior and intent of security breaches sooner by providing complex analysis prior 
to a full-blown malware attack. 
\item Strengthen security defenses and prevent malicious software from running with state-of-the-art technology 
called whitelisting. 
\item Take action sooner when security breaches occur; unlike other competitors, Cerberus doesn't rely on a sandbox 
or signature-based solutions. 
\end{itemize}

\section{Proposed Solution}

As can be seen from the existing alternative software, collaboration is a major feature included in all these programs, and while Autopsy also allows the software to be
configured in a manner to allow collaboration it involves a complicated setup and has certain limitations.
The setup required for collaborative cases in Autopsy is as follows:
\begin{itemize}
 \item Shared hard drive accessible for every computer setup for every machine using the same drive letter
 \item PostgreSQL Database server 
 \item Solr indexing server
 \item ActiveMQ messaging server
\end{itemize}
The limitations of Autopsy's multi-user case feature are that it requires every user to be using a Windows O.S. computer and it also requires specific configuration on each
machine involved in the case that is set up.

The proposed solution aims to cover three main aspects:
\begin{enumerate}
 \item Accessibility
 \item Collaboration
 \item Organization 
\end{enumerate}

\subsection{Accessibility}

Given a client-server architecture, any client with access to the network where the server is located can access the contents provided by the server.
The aim is to condense the processing heavy features of Autopsy in a single server and provide any number of clients access to this information, requiring less 
resources from each client, allowing collaboration and removing any type of setup required for each of the client machines, while also providing a more modern 
and user friendly design.

\subsection{Collaboration}

In order to provide collaboration, all the information is maintained in a single server, or a collection of servers providing different functions (like exposing endpoints,
storing data, and indexing searches), and every client can preform all the allowed actions whether they consist in consulting, generating, or removing information. 
Collaboration comes naturally with a client-server model, as the same server that provides the endpoints can also communicate with each client using websockets, and maintain 
information in a coordinated state along every connected client.

\subsection{Organization}

Digital forensics investigations are usually done by specialized organizations, who need to organize their human resources in an efficient and secure manner.
Assigning investigators to teams, assigning teams to cases, allowing access to the platform and certain information is a critical part of the activities preformed by
a company that specializes in digital forensics, so having these functionalities properly integrated into a digital forensics platform should be an important feature.