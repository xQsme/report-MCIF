\addtocontents{toc}{\protect\vspace{\beforebibskip}} % Place slightly below the rest of the document content in the table

%************************************************
\chapter{Introduction}
\label{ch:introduction}
%************************************************

\section{Digital Forensics}

\subsection{Contextualization}

Forensic science is the use of scientific methods or expertise to investigate crimes
or examine evidence that might be presented in a court of law.

The definition of digital forensics is directly related to the definition of
computer forensics which is the collection, preservation, analysis,
and presentation (\citetitle{daniels}) of evidence stemming from digital sources for use in a legal matter
using investigative processes, tools, and practices.

Digital forensics is the application of computer technology to criminal cases where evidence
includes items that are created by digital systems.

Digital forensics, according to \citeauthor{nist}, is the field of forensic science that is concerned with retrieving,
storing and analyzing electronic data that can be useful in criminal investigations.
This includes information from computers, hard drives, mobile phones and other data
storage devices.

Digital forensic investigators face challenges such as extracting data from damaged or destroyed
devices, locating individual items of evidence among vast quantities of data,
and ensuring that their methods capture data reliably without altering it in any way.

Personal data should ultimately be attributable to an individual; however, making
that attribution can be difficult due to the presence or absence of individualized
user accounts, security to protect those user accounts, and the actual placement of a
person at the same location and time when the data is created.

\begin{center}
A PARTIR DAQUI TODO O CONTEUDO DA INTRODUCAO É COPIA INTEGRAL DE FONTES EXTERNAS
\end{center}


\subsection{Digital Evidence}

Digital evidence begins as electronic data, either in the form of a transaction, a document,
or some type of media such as an audio or video recording. Transactions
include financial transactions created during the process of making a purchase, paying
a bill, withdrawing cash, and even writing a check. While writing a check might
seem to be an old-fashioned method that is not digital or electronic in nature, the
processing of that written check is electronic and is stored at your bank or credit
card company. Nearly every kind of transaction today is eventually digitized at
some point and becomes digital evidence: doctor visits, construction projects,
getting prescriptions filled, registering a child at daycare, and even taking the pet in
for a rabies shot.

In today’s connected world, it is nearly impossible to be completely “off the
net” such that your activities do not create some form of electronic record.

The explosion of social media sites has created a whole new area of electronic
evidence that is both pervasive and persistent. People today are sharing their everyday
activities, their thoughts, their personal photos, and even their locations via
social media such as Twitter, Facebook, and MySpace. Add to this the explosion
of the blogosphere, where individuals act as citizen journalists and self-publish
blog posts on the Internet ranging from their political views to their personal family
blogs with pictures of their kids and pets.

In order for electronic data to become digital evidence, it must be stored some-
where that is ultimately accessible in some fashion; and it must also be recoverable
by a forensic examiner. One of the great challenges today is not whether digital
evidence may exist, but where the evidence is stored, getting access to that storage,
and finally, recovering and processing that digital evidence for relevance in light of
a civil or criminal action.

The potential storage options for electronic evidence are expanding every
day, from data stored on cell phones and pad computers to storage in the “cloud”
where a third-party service provides hard drive space on the Internet for people and
businesses to store data.

More and more everyday computing processes are moving to the Internet
where companies offer software as a service. Software as a service means that
the customer no longer has to purchase and install software on their computer.
Some examples of software as a service range from accounting programs like
QuickBooks Online, Salesforce.com, or a sales management application to online
games that are entirely played via the Internet with no required software installation
on the local computer.

\subsection{Processes and Procedures}

Digital forensics is the application of forensic science to electronic evidence in a
legal matter.

While there are many different subdisciplines and many types of devices, communication,
and storage methods around today, the basic tenets of digital forensics
apply to all of them.

These tenets encompass four areas:

\begin{enumerate}
\item Acquisition
\item Preservation
\item Analysis
\item Presentation
\end{enumerate}

Each of these areas includes specific forensic processes and procedures.

\subsubsection{Acquisition}

Acquisition is the process of actually collecting electronic data. For example, seizing
a computer at a crime scene or taking custody of a computer in a civil suit is
part of the acquisition process. Making a forensic copy of a computer hard drive
is also part of the acquisition process. In the digital forensics field, examiners refer
to making these forensic copies of evidence as “acquiring” a hard drive rather than
copying a hard drive. This is to avoid the confusion that could be caused by using
the term “copy,” since making a copy of something does not imply that the copy
was made in a forensically sound manner.

Acquisition is the first step in the forensic process and is critical to ensure the
integrity of the evidence. As acquisition is the first contact with the evidence, it is
the point where evidence is most likely to be damaged or destroyed. Simply turning
on a computer can lead to the modification of hundreds of evidentiary items
including files, date and time stamps, introduction of new Internet history, and
the destruction of files that could be recovered from areas of the hard drive that
are in the area of unallocated space (see the section “Deleted File Recovery” in
Chapter 29).

\subsubsection{Preservation}

As evidence is collected, it must be preserved in a state that is defendable in court.
Preservation is the process of creating a chain of custody that begins prior to collection
and ends when evidence is released to the owner or destroyed. Any break
in the chain of custody can lead to questions about the validity of the evidence.
Additionally, preservation includes keeping the evidence safe from intentional
destruction by malicious persons or accidental modification by untrained personnel.

A chain of custody log best illustrates an example of preservation. Chain
of custody logs should include every instance that a piece of evidence has been
touched, including the initial collection of the device storing the evidence, the
transport and storage of the evidence, and any time the evidence is checked out for
handling by forensic examiners or other personnel. At no time should there be a
break in this chain.

\subsubsection{Analysis}

Analysis is the process of locating and collecting evidentiary items from evidence
that has been collected in a case. In a case involving spousal infidelity, the evidence
that must be located can include e-mails and chat logs between the spouse and the
paramour. In a fraud case, financial records would be the target of the analysis, as
well as the possible deletion of records involving financial transactions. In a child
pornography case, locating contraband pictures and movies would be the target of
the examination. Each case is unique in this respect as the circumstances surrounding
each case can vary widely, not only in the evidence being sought, but also in the
approach used to perform the analysis. The analysis portion is also the area where
the individual skills, tools used, and the training of the forensic examiner have
the greatest impact on the outcome of the examination. For information on forensic
tools used by examiners, see Chapter 5. Considering that electronic evidence
appears in so many forms and comes from so many disparate locations and devices,
the training and experience of the examiner begins to have an ever-greater impact
on the success of the examination.

The analysis phase is also where the greatest disparity begins to become a factor
between the skills and approach of a “computer expert” and those of a computer or
digital forensics expert. While a computer expert may understand many aspects of
computer usage and data, a properly trained forensic expert will be well versed in
recovering data as well as in proper examination techniques.

Analysis of digital evidence is more than just determining whether something
like a file or e-mail message exists on a hard drive. It also includes finding out how
that file or e-mail message got on the hard drive, and if possible, who put the file or
message on the hard drive.

\subsubsection{Presentation}

Presentation of the examiner’s findings is the last step in the process of forensic
analysis of electronic evidence. This includes not only the written findings or forensic
report, but also the creation of affidavits, depositions of experts, and court testimony.
There are no hard and fast rules or standards for reporting the results of an
examination. Each agency or private entity may have its own particular guidelines
for reporting. However, forensic examination reports should be written clearly, concisely,
and accurately, explaining what was examined, the tools used for the examination,
the processes used by the examiner, and the results of that examination.
The report should also include the collection methods used, including specific steps
taken to protect and preserve the original evidence and how the verification of the
evidence was performed.

In general, a digital forensics report should include:

\begin{itemize}
\item Background and experience of the examiner
\item Tools used in the examination
\item Methods used to verify the data
\item Processes used to recover and extract the data
\item Statement of what the examiner found
\item Actual data recovered to support the statement of findings.
\end{itemize}

\section{The Sleuth Kit}

The Sleuth Kit (TSK) is a library and collection of command line tools that allow
you to investigate disk images. The core functionality of TSK allows you to analyze
volume and file system data. The plug-in framework allows you to incorporate
additional modules to analyze file contents and build automated systems. The
library can be incorporated into larger digital forensics tools and the command
line tools can be directly used to find evidence.

Description

The original part of Sleuth Kit is a C library and collection of command line
file and volume system forensic analysis tools. The file system tools allow you
to examine file systems of a suspect computer in a non-intrusive fashion. Because
the tools do not rely on the operating system to process the file systems, deleted
and hidden content is shown. It runs on Windows and Unix platforms.

The volume system (media management) tools allow you to examine the layout of disks
and other media. The Sleuth Kit supports DOS partitions, BSD partitions 
(disk labels), Mac partitions, Sun slices (Volume Table of Contents), and 
GPT disks. With these tools, you can identify where partitions are located 
and extract them so that they can be analyzed with file system analysis tools.

When performing a complete analysis of a system, we all know that command line 
tools can become tedious. Autopsy is a graphical interface to the tools in The 
Sleuth Kit, which allows you to more easily conduct an investigation. Autopsy 
provides case management, image integrity, keyword searching, and other automated
operations.

A complete analysis also requires more than just file and volume system analysis.
However, a single tool can't provide support for all file types and analysis 
techniques. The TSK Framework allows tool so easily incorporate file analysis
modules that were written by other developers. If you are developing a tool,
consider incorporating in the framework or developing your analysis technique
as a module into the framework.

Input Data

Analyzes raw (i.e. dd), Expert Witness (i.e. EnCase) and AFF file system and disk
images.
Supports the NTFS, FAT, ExFAT, UFS 1, UFS 2, EXT2FS, EXT3FS, Ext4, HFS, ISO 9660,
and YAFFS2 file systems (even when the host operating system does not or has a
different endian ordering).
Tools can be run on a live Windows or UNIX system during Incident Response. 
These tools will show files that have been "hidden" by rootkits and will not 
modify the A-Time of files that are viewed.

Search Techniques

List allocated and deleted ASCII and Unicode file names.
Display the details and contents of all NTFS attributes (including all Alternate
Data Streams).
Display file system and meta-data structure details.
Create time lines of file activity, which can be imported into a spread sheet to 
create graphs and reports.
Lookup file hashes in a hash database, such as the NIST NSRL, Hash Keeper, and 
custom databases that have been created with the 'md5sum' tool.
Organize files based on their type (for example all executables, jpegs, and
documents are separated). Pages of thumbnails can be made of graphic images 
for quick analysis.

The Sleuth Kit is written in C and Perl and uses some code and design from 
The Coroner's Toolkit (TCT). The Sleuth Kit has been tested on:

\begin{itemize}
\item Linux
\item Mac OS X
\item Windows (Visual Studio and mingw)
\item CYGWIN
\item Open \& FreeBSD
\item Solaris
\end{itemize}

\section{Autopsy}

Autopsy is a digital forensics platform and graphical interface to The Sleuth Kit
and other digital forensics tools. It is used by law enforcement, military, 
and corporate examiners to investigate what happened on a computer. You can even 
use it to recover photos from your camera's memory card.

Easy to Use
Autopsy was designed to be intuitive out of the box. Installation is easy and
wizards guide you through every step. All results are found in a single tree.
See the intuitive page for more details.

Extensible
Autopsy was designed to be an end-to-end platform with modules that come with
it out of the box and others that are available from third-parties. Some of the
modules provide:

\begin{itemize}
\item Timeline Analysis - Advanced graphical event viewing interface (video 
tutorial included).
\item Hash Filtering - Flag known bad files and ignore known good.
\item Keyword Search - Indexed keyword search to find files that mention relevant
terms.
\item Web Artifacts - Extract history, bookmarks, and cookies from Firefox, Chrome,
and IE.
\item Data Carving - Recover deleted files from unallocated space using PhotoRec
\item Multimedia - Extract EXIF from pictures and watch videos.
\item Indicators of Compromise - Scan a computer using STIX.
\item See the Features page for more details. Developers should refer to the
module development page for details on building modules.
\end{itemize}

Fast
Everyone wants results yesterday. Autopsy runs background tasks in parallel using multiple cores and provides results to you as soon as they are found. It may take hours to fully search the drive, but you will know in minutes if your keywords were found in the user's home folder. See the fast results page for more details.

Cost Effective
Autopsy is free. As budgets are decreasing, cost effective digital forensics solutions are essential. Autopsy offers the same core features as other digital forensics tools and offers other essential features, such as web artifact analysis and registry analysis, that other commercial tools do not provide.

\section{Alternative Software}

%%https://en.wikipedia.org/wiki/List_of_digital_forensics_tools

\subsection{nuix}

%%https://www.nuix.com/who-we-are

\subsection{OSForensics}

%%https://www.osforensics.com/

\section{Proposed Solution}

Why is the project being developed???

\pagebreak

Este documento serve de orientação para o relatório da unidade curricular de Projecto Informático do Curso de Engenharia Informática da ESTG – IPLEIRIA. Como tal, é constituído por um conjunto predefinido de estilos a utilizar. Estes estilos devem ser utilizados sem serem alterados ou substituídos. Para começar facilmente a escrever o relatório, basta guardar uma cópia deste documento e substituir os campos e as secções de acordo com o projecto em questão.

Embora possa parecer uma abordagem demasiadamente descritiva para a escrita do relatório, as intenções pretendidas com este documento são:

\begin{itemize}
 \item Focar os alunos na produção de conteúdos com qualidade, em vez de se preocuparem com formatações de tipos de letra, parágrafos, etc.;
 \item Ao fornecer um documento de orientação de estilos a Escola beneficia de um aspecto profissional e consistente da globalidade dos seus relatórios de projecto.
\end{itemize}


Quanto ao conteúdo de uma introdução, ele deve preparar o leitor para o resto do relatório. Deve conter o detalhe suficiente para que alguém das áreas de conhecimento envolvidas possa entender o assunto do trabalho. A maior parte das introduções contêm três partes para fornecer contexto ao trabalho: objectivos, âmbito e background do trabalho do projecto. Estas partes muitas vezes sobrepõem-se, e podem por vezes ser omitidas simplesmente porque não faz sentido incluir alguma delas.

É de extrema importância considerar os objectivos do trabalho e do relatório na introdução. Se os autores não entenderem bem os objectivos do trabalho, dificilmente o leitor os entenderá. As seguintes questões ajudam a pensar nos objectivos do trabalho e na razão da escrita do relatório:

\begin{enumerate}
 \item O que foi descoberto ou provado?
 \item Em que tipos de problemas se trabalhou?
 \item Porque é que se trabalhou nestes problemas? Se o problema lhe foi atribuído, deve tentar-se saber as razões pelas quais os orientadores o formularam, e o que era suposto que os alunos aprendessem ao trabalharem neste problema;
 \item Qual a razão da escrita deste relatório?
 \item O que é que o leitor deve ficar a saber quando acabar de ler este relatório?
\end{enumerate}


O âmbito deve indicar as áreas de conhecimento envolvidas e realçar a metodologia utilizada no trabalho de projecto. Referir o âmbito do projecto na introdução ajuda o leitor a perceber os parâmetros de entrada do trabalho e do relatório, bem como a identificar as principais restrições consideradas (por exemplo “existem 5 Sistemas Operativos para trabalhar com determinado hardware, mas somente 3 foram considerados neste estudo”). As seguintes questões ajudam a pensar no âmbito do trabalho e do relatório:

\begin{enumerate}
 \item De que forma foi abordado o problema, e qual a razão para tal abordagem?
 \item Existiam outras abordagens óbvias que se poderiam ter adoptado ? Que limitações impediram que se tentassem outras abordagens?
 \item Que factores contribuíram para a escolha da forma de como se abordou o problema, e qual o mais relevante nessa escolha?
\end{enumerate}

A informação de background inclui os conhecimentos que o leitor deve possuir por forma a compreender o trabalho de projecto e correspondente relatório. Estes conhecimentos incluem a percepção de trabalhos prévios que motivaram a proposta do projecto corrente, ou referências a trabalhos teóricos e práticos relacionados com os objectivos e âmbito descritos acima. Devem remeter-se para anexos documentos que poderão ajudar na percepção de teorias, metodologias, técnicas ou ferramentas utilizadas no trabalho de projecto. As seguintes questões ajudam a pensar no background necessário para o trabalho e para o relatório:

\begin{enumerate}
 \item Que factos deve o leitor conhecer para perceber o relatório?
 \item Porque é que o projecto foi autorizado ou atribuído?
 \item Quem já fez trabalho prévio para resolver o problema colocado pelo projecto?
\end{enumerate}

Por fim, a introdução deve descrever como foi organizado o relatório, referindo brevemente o propósito de cada secção considerada no mesmo.

O resto deste documento dá uma breve perspectiva das partes seguintes que devem constar do relatório, bem como de outros aspectos de formatação.