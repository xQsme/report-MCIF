\addtocontents{toc}{\protect\vspace{\beforebibskip}} % Place slightly below the rest of the document content in the table


%Kensentme platform delevopment


%************************************************
\chapter{Internship Programme}
\label{ch:development}
%************************************************

The internship started on September 2nd 2019, being planned to last for 9 months, ending on the 29th of May 2020.

The first day at the company was interesting, meeting most of the personalities of the workplace and getting set up with my work environment,
consisting of a desk, chair, computer and 2 monitors.

As soon as the computer was properly configured with all the software needed the project became the main focus of my time at the company,
making progress daily, sometimes just planning the architecture or analyzing the source code, sometimes just writing code, and sometimes a mix of both.

\section{Project Planning}

Autopsy by itself is capable of providing a distributed solution for multi user collaboration, but it is very resource intensive and requires many complex configuration steps,
and is also only fully supported on the Windows O.S.

The plan for this internship is to achieve the same kind of functionality provided by the original software, but without the dependency on harduous pre-configuration
or hardware intensive requirements, resulting in needing only single capable server, and allowing the program to be used by multiple low capacity client devices using 
any web capable O.S.

To achieve that goal, the project will be an adaptation of the original Autopsy source code, into a client-server model, with the server developed in Java using the Quarkus framework,
and the client developed in Javascript using the React framework.

The project is outlined to work in a multi user environment, allowing users to be assigned to teams and teams assigned to cases, and allowing multiple users to interact with
a case simultaneously.

Autopsy has a major limitation, which is it only allows one case to be open at a time, ideally in this project we should find a workarround to allow working on multiple cases
at once, but we feel like dedicating a server instance (which a single machine can have many virtualized within) per case is a good enough approach, though as future work the
ability to spawn different containers as requested to work on different cases at once is an interesting challenge.

Given that the core features will be running in a remote server, it was decided that the addition of data sources to cases will be handled by an FTP client, allowing users
to transfer files to their respective data source directories, and FTP access will be controlled according to each user's credentials on the platform.

\section{Autopsy source code analysis}

Autopsy is a digital forensics analysis software that is available as Open Source Software on GitHub.

With the goals set for this project, the source code was analyzed to understand which components need to be replicated and adapted in order to obtain the same logic flow.

The ``Core`` module is where the most important components are located, and after analysis it was concluded that the following directories contain relevant information:

\begin{itemize}
 \item Actions: user interactions 
 \item Casemodule: case class and other resources needed for the functioning of an autopsy case like data sources and artifacts
 \item Centralrepository: data persisted and accessed by multiple cases (Correlation Engine)
 \item Contentviewers: panels used for data representation 
 \item Coordinationservice: configuration information distribution system
 \item Core: addition of command line options, system configurations and collaboration monitor 
 \item Corecomponents: main user interface components
 \item Datamodel: all the entities needed to represent ingested data
 \item Datasourceprocessors: data processing utilities 
 \item Directorytree: file explorer for ingested artifacts
 \item Ingest: utilities and events for data ingestion 
 \item Keywordsearchservice: utility to search artifacts by keyword 
 \item Modules: all the pre-included modules (data ingestion procedures)
 \item Progress: progress indicators and similar classes
 \item Python: resources needed for the functioning of the Jython language
 \item Rejview: resources used to analyse Windows registry
 \item Report: report generator
 \item Timeline: recent addition to Autopsy, allows visualization of artifacts in temporal chart, only available for Windows O.S.
\end{itemize}

The ``KeywordSearch`` module is also of critical importance as it provides one of the most meaningful features which is filtering all the artifacts in a case with a keyword
search using the Apache Solr search platform, which indexes the text contents of all the artifacts and allows extremely fast searching through a large amount of data.  

Another module that needs to be adapted is the ``RecentActivity`` module, which contains the tools needed to extract information from browsers, providing a great amount of
critical evidence from data sources.

\section{Development}

\subsection{Management Entities}

As a first step in the development, the different persisted entities were created, which are Users, Teams, and Cases. All the endpoints for actions involving these 
entities were created, resulting in the ability for the client program to interact with these entities and modify their relationships and other variables.
For these functionalities there are two roles associated, the Manager role allows manipulation of the existing entities while the Investigator role only has access to his
own information and the teams and cases he was assigned to. For these interactions, it was decided to create a drag and drop interface, which allows users to be dragged into
teams and teams dragged into cases. All these entities are listed side by side and each have their own options and filtering input, as can be observed in Figure \ref{fig:users}.

\begin{figure}[h]
 \centering
 \includegraphics[width=1\linewidth]{imgs/users.png}
 \caption{Entity Management Interface}
 \label{fig:users}
\end{figure}

Users can change their own profile picture, while Managers can also change any team or case's display picture.
Managers can add new users to the platform, can approve membership requests, can enable/disable user accounts and can create new teams.
When a user is added by a Manager or his membership request is approved he must define a password while activating his e-mail account.

\subsection{Basic Autopsy Functionalities}

Then the most basic functionality from Autopsy was adapted, the ability to open an autopsy case. For this some elements of the original Casemodule package were adapted,
and after that all the other similar actions like closing, creating and deleting cases were also adapted.

Autopsy cases have a case file containing case metadata, which allows the program to connect to the right database when the case is open, this database is also present
in a file inside the filesystem, which uses the SQLite database engine, so for the cases to be usable in the server these files must also be present in the server,
which resulted in the creation of a directory within the server called ``repository``, containing all the different cases created within the application.

Later in the development there was the need to create an additional directory alongside the ``repository`` called ``central-repository``
which contains the database used by the Correlation Engine to ingest data that can be queried by any case.

\subsection{Ingested Results Presentation}

Ingested results are the items present inside the provided data sources, Autopsy runs multiple modules on each data source and extracts these results using The Sleuth Kit,
extracted results can either be a file instance or an artifact (which is something that corresponds only to a piece of information inside a file).

The ingested results are presented in three different containers, one taking the shape of a file explorer, allowing exploration of the structure of all the results, 
one taking the shape of a table or thumbnail viewer, presenting all the contents of the results selected from the explorer, and one taking the shape of a content viewer, allowing visualization
of the data contained inside the result selected from the table as can be seen in Figure \ref{fig:data}.

\begin{figure}[h]
 \centering
 \includegraphics[width=1\linewidth]{imgs/data.png}
 \caption{Ingested Results Presentation}
 \label{fig:data}
\end{figure}

The content viewer can display different kinds of information depending on the type of item selected, these can be some of the following:
\begin{itemize}
 \item Text browser
 \item Media viewer
 \item Database browser
 \item Registry browser
 \item Key-value browser
 \item Table data viewer
\end{itemize}

The layout for the ingested results presentation, and for all the case related actions, was based on the original Autopsy layout, so that each container can be re-sized as
needed, allowing the user to focus on the information that is most important to him.

\subsection{Data Sources}

Using the same credentials used to log-in to the platform, the user can also upload data source files into his folder located inside the server, using an FTP client like filezilla.
Then the user can browse these directories using the web interface and select a data source to add to the case, as can be seen in Figure \ref{fig:datasource}

\begin{figure}[h]
 \centering
 \includegraphics[width=0.85\linewidth]{imgs/data-sources.png}
 \caption{Data Source Selection}
 \label{fig:datasource}
\end{figure}

The procedure for adding a data source to a case was adapted from Autopsy's source code, and depends on the type of data source added, which can be one of the following:
\begin{itemize}
 \item Disk image
 \item Virtual machine
 \item Logical file collection
 \item Unallocated space image file
 \item Autopsy logical imager results
 \item Memory image files from Volatility
\end{itemize}

Local disk data sources were also an option provided by Autopsy but since the local disks the software has access to belong to a server, this feature is undesired.

\subsection{Data Ingestion Modules}

Firstly the default modules included with Autopsy were adapted, and can be ran through the web interface. When the modules are running the server communicates to each client when
they start, updates their progress, and informs that they finished, as can be seen in Figure \ref{fig:modules}.

\begin{figure}[h]
 \centering
 \includegraphics[width=0.5\linewidth]{imgs/modules.png}
 \caption{Ingest Modules Communication}
 \label{fig:modules}
\end{figure}

When a module finishes the case explorer is updated with the extracted information.

\subsection{Report Modules}

TODO

\section{Other Projects}

TODO
